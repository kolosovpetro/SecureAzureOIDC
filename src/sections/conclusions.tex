In this manuscript we explore the problem of secure storage and transfer of access tokens between microservices.
Particular attention was paid to possible vulnerabilities during transfer of access tokens such
as Cross-Site Scripting (XSS) and Cross-Site Request Forgery (CSRF).

To eliminate these vulnerabilities, it is necessary to store authorization tokens in cookies with mandatory
\texttt{HttpOnly} and \texttt{SameSite} settings such that \texttt{SameSite} values should be \texttt{Lax} or \texttt{Strict}.
Therefore, cookies are either transmitted via secure \texttt{HTTP} methods or not transmitted at all.

User authentication to be implemented using the OIDC protocol~\cite{siriwardenaOpenid2020, sakimuraOpenid2014}
in couple with Authorization code flow with PKCE~\cite{bradley2015rfc}.
The main principle of the OIDC protocol is described more detailed in Chapter 2.

Also, we provide an authentication / authorization implementation based on the ASP.NET Core Web API backend and Angular
frontend application.
These apps are stored under single domain to eliminate necessity to transfer authorization cookies cross domain way.
The transfer of access tokens to microservices is implemented using Reverse Proxy YARP~\cite{microsoftYarp2021} so that
the access token is automatically substituted in the request header.

In addition, we proposed a mechanism to refresh an access token through the Ticket Store~\cite{microsoftIticketstore2023} entity
and Hosted Service~\cite{microsoftHostedservice2023}.
Therefore, the Ticket Store checks each request for access token expiration.
In case of expiration of the access token, the access token is refreshed by means of authorization microservice.
Ticket store also stores the pair of the access and refresh token inside \texttt{AuthenticationTicket} entity~\cite{microsoftAuthenticationTicket2023}.

Finally, in this manuscript we proposed a solution to the problem of securely storing an access token and passing it between microservices,
eliminating the Cross-Site Scripting (XSS) and Cross-Site Request Forgery (CSRF) vulnerabilities.