In this manuscript we explore the problem of secure storage and transfer of access tokens between microservices.
Particular attention was paid to possible vulnerabilities during transfer of access tokens such
as Cross-Site Scripting (XSS) and Cross-Site Request Forgery (CSRF).

To eliminate these vulnerabilities, it is necessary to store authorization tokens in cookies with mandatory
\texttt{HttpOnly} and \texttt{SameSite} settings such that \texttt{SameSite} values should be \texttt{Lax} or \texttt{Strict}.
Therefore, cookies are either transmitted via secure \texttt{HTTP} methods or not transmitted at all.

Authentication to be implemented using the OIDC protocol~\cite{siriwardenaOpenid2020, sakimuraOpenid2014}
and Authorization code flow with PKCE~\cite{bradley2015rfc}.
The main principle of the OIDC protocol is described more detailed in Chapter 2.

Also, we provide an authentication / authorization implementation based on the ASP.NET Core Web API backend and Angular
frontend application.
These applications are stored under the single domain to eliminate necessity to transfer authorization cookies cross domain way.
Transfer of access tokens between microservices is implemented using Reverse Proxy YARP~\cite{microsoftYarp2021} so that
the access token is automatically substituted in the request header.

In addition, we proposed a mechanism to refresh an access token through the
\texttt{TicketStore}~\cite{microsoftIticketstore2023} entity
and \texttt{HostedService} class~\cite{microsoftHostedservice2023}.
Therefore, the \texttt{TicketStore} checks each request for access token expiration.
In case of expiration of the access token, the access token is refreshed by means of authorization microservice.
The \texttt{TicketStore} also stores pairs of the access and refresh token inside
\texttt{AuthenticationTicket} entity~\cite{microsoftAuthenticationTicket2023}.

Finally, in this manuscript we proposed a solution to the problem of securely storing an access token and passing it between microservices,
eliminating the Cross-Site Scripting (XSS) and Cross-Site Request Forgery (CSRF) vulnerabilities.