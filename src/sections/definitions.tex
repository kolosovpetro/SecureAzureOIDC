\begin{itemize}
    \item \textbf{Access Tokens} are credentials used to access protected resources.
    An access token is a string representing an authorization issued to the client.
    The string is usually opaque to the client.
    Tokens represent specific scopes and
    durations of access, granted by the resource owner, and enforced by the resource server and authorization server.
    Each part is Base64-encoded.
    \item \textbf{Refresh Tokens} are credentials used to obtain access tokens.
    Refresh tokens are issued to the client
    by the authorization server and are used to obtain a new access token when the current access token becomes invalid or expires,
    or to obtain additional access tokens with identical or narrower scope (access tokens may have a shorter lifetime and
    fewer permissions than authorized by the resource owner).
    \item \textbf{Resource Owner} is an entity capable of granting access to a protected resource.
    When the resource owner is a person, it is referred to as an end-user.
    \item \textbf{Resource Server} is the server hosting the protected resources, capable of accepting and responding to protected
    resource requests using access tokens.
    \item \textbf{Client} is an application making protected resource requests on behalf of the resource owner and with its authorization.
    This term does not imply any particular implementation characteristics (e.g., whether the application executes on a server,
    a desktop, or other devices).
    \item \textbf{Authorization Server} is the server issuing access tokens to the client after successfully authenticating the resource owner
    and obtaining authorization.
\end{itemize}